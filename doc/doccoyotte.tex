%        File: doccoyotte.tex
%     Created: Mar jui 28 08:00  2011 C
% Last Change: Mar jui 28 08:00  2011 C
%
\documentclass[a4paper]{article}
\usepackage[french]{babel}
\usepackage[utf8]{inputenc}
\title{Coyotte user manual}
\author{Hervé Rouault}
\begin{document}
\maketitle

\section{Plugging}
Several components have to be plugged before use~:
\begin{enumerate}
   \item Plug first the USB cable on the computer (any port)
   \item Launch Cygwin (icon on the desktop) and type in~: \textsf{screen /dev/tty.S4} (S4 can be S5 depending on the port you plugged the USB cable in). You should then see several columns of numbers, one of them is the current temperature.
   \item Plug the temperature sensor (purple cable), this is not polarized and hence corresponds to the two wires of identical color.
   \item Plug the Peltier module (green and purple wires).
   \item Plug the +12V power adaptor
\end{enumerate}

\section{Keyboard control}
To control the temperature, you have to use the keyboard~:

\begin{tabular}{llll}
   Key & Function\\
   \hline
   ``p'' & ON/OFF\\
   ``+/-'' & 0.5 target temperature increment\\
   ``a'' & Target=18\degre C\\
   ``b'' & Target=20\degre C\\
   ``c'' & Target=21\degre C\\
   ``d'' & Target=25\degre C\\
   ``e'' & Target=29\degre C\\
\end{tabular}

\section{Advanced control}
The PID parameters can be tuned in real time~:

\begin{tabular}{llll}
   Key & Function\\
   \hline
   ``u/j'' & P increment\\
   ``i/k'' & I increment\\
   ``o/l'' & D increment\\
\end{tabular}


\end{document}


